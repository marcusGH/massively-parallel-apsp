\documentclass[11pt]{article}
\usepackage{parskip,xspace}
\usepackage[margin=3cm]{geometry} 
\usepackage{array} 
\usepackage{natbib}
\usepackage[colorlinks,citecolor=red,urlcolor=blue,bookmarks=false,hypertexnames=true]{hyperref} 
\usepackage{longtable} 
\usepackage{setspace} \onehalfspacing


\def\authorname{Marcus A.\ K.\ September\xspace}
\def\authorcollege{Clare College\xspace}
\def\authoremail{maks2@cam.ac.uk}
\def\dissertationtitle{A parallel algorithm for all-pairs shortest paths that minimises data movement}
\title{
    {\large Part II Computer Science Progress Report} \\~\\
    \dissertationtitle}
\author{
    \authorname, \authorcollege \\
    \texttt{\authoremail}
}

\begin{document}
\maketitle

\paragraph{Project supervisor:}%
\label{par:Project Supervisor}
Dr J. Modi

\paragraph{Director of studies:}%
\label{par:Director of Studies:}
Prof L. C. Paulson

\paragraph{Project overseers:}%
\label{par:Project Overseers:}
Prof R. Mantiuk \& Prof A. M. Pitts

\newpage

% Introduction XXX: need more work
\section*{Progress report}

The project is currently on schedule, and work items have been completed as
they were described on the timetable. There has not been any unexpected
difficulties with implementation. Additionally, all the milestones of the work
items leading up to, but not including, evaluation have been met. I am
currently doing evaluation, which is in line with the timetable as the period
20 Jan -- 2 Feb is set of for this.

I have implemented a python script and Java class that, respectively, downloads
and transforms various graph datasets into the appropriate format to feed into
the all-pairs shortest paths (APSP) algorithm.  I have also finished the
multiprocessor simulation; It now provides the programmer with an interface
where they can program what a general processing element (PE) $PE(i, j)$ should
do during its computation and communication phases, having access to methods
such as \verb|send(other_i, other_j, data)| and \verb|broadcastRow(data)|. This
description can then be passed onto a \texttt{Manager} which instantiates
\texttt{Worker}s according to the description, loads the initial memory
content, and runs them until they have completed a specified number of work
phases.  It also handles their communication using a \texttt{MemoryController}
that I have implemented.  I have also implemented the \texttt{FoxOtto}
algorithm, which runs on the multiprocessor simulator.  Additionally, I have
finished implementing the main APSP algorithm, which uses \texttt{FoxOtto}
min-plus matrix multiplication as a subroutine.  I have also thoroughly tested
the whole algorithm for correctness by manually creating small example graphs
and verifying that the shortest paths found are correct, and I have also run it
on a large graph with 250 nodes and compared the results with the output of a
serial Dijkstra algorithm I am certain is correct.

Currently, I am working on the evaluation of the algorithm. I have implemented
timing and communication-counting wrappers, and used these to estimate the
computation and communication time required by the algorithm. I have also been
researching how the algorithm implemented would map onto real multiprocessor
hardware, and looked into what the communication latency and bandwidth are on such
hardware. This will be used to create a more realistic estimate of the
communication time. What remains to be done as part of evaluation is to do more
measurements for graphs of various sizes to determine how the
algorithm's performance scales.

Going forward, I will finish evaluation, then aim to complete two of my
extensions, one of which I'm more than halfway done with: Graph compression
through removal of 2-degree nodes.  The other extension is generalizing the
implementation to allow each PE to handle a sub-matrix of size greater than $1
\times 1$.  Some weeks of February are set off for extensions, so this work will
be according to the initial work plan.  


\end{document}
