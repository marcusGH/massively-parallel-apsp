\documentclass{beamer}
\usepackage[]{geometry,array} 
\usepackage[]{longtable} 

\usetheme{default} % Copenhagen for the cool blocks
\def\dissertationtitle{A parallel algorithm for all-pairs shortest paths that minimises data movement}

\title{Progress presentation}
\subtitle{\dissertationtitle}
\author{Marcus A.\ K.\ September}
\date{\today}

\begin{document}

\begin{frame}
    \titlepage
\end{frame}

\begin{frame}
    \frametitle{Introduction}
    % TODO: explain what the project is about here, because overseers and fellow
    %       students will have forgotten it, not know it
    \begin{block}{What is the project about?}
        \begin{itemize} \item Develop a parallel algorithm for solving
            all-pairs shortest paths (APSP)
                % This will use matrix multiplication as a subroutine
            \item A 2D grid of processing elements and message passing
                % Each has some private memory and can send messages to each
                % other
            \item Multiprocessor simulation
            \item Parallelise a matrix multiplication, Minimise data movement with \texttt{FoxOtto}
                % Used as a subroutine ...
                % To implement it on..... ? TODO: rephrase and restructure
        \end{itemize}
    \end{block}
\end{frame}

\begin{frame}{Current state of progress}
    \begin{block}{Timetable}
        \begin{longtable}{m{40pt}m{75pt}m{120pt}}
            $\dots$ & $\dots$ & $\dots$ \\
            \hline
            16 -- 17 & 20 Jan -- 2 Feb & \textbf{Evaluation} $\cdots$ \\
            \hline
            18 -- 20 & 3 Feb -- 23 Feb & \textbf{Extensions} $\cdots$ \\
            \hline
            21 -- $\cdots$ & 24 Feb -- $\cdots$ & \textbf{Dissertation writing} \\
        \end{longtable}
    \end{block}
    \begin{block}{Current state}
    \begin{itemize}
            % TODO: explain this: When started doing evaluation, discovered
            % some parts missing: some more functionality needed by the timing
            % analyser that reads the timing data found. Additionally,
            % difficult to get sufficiently different problem sizes in fine
            % enough granularity with the real-world datasets I am using, so
            % considering using random graph generation with networkX in
            % python. These two items will probably not take too long, and I
            % have already completed 1.5 of my extensions... Compare with
            % timetable...

            % In the timetable, I said I would be done with evaluation on 2 Feb,
            % but there is still some work left to be done here. I have a script to
            % extract subgraphs of real-world datasets I am using, but it's unreliable
            % to control the problem size with this method. Need to make more
            % measurements to get a sense of scaling, so a work item remaining is
            % generating random graphs to do more measurements on execution time. Already found a library
            % to do this, and plan on setting parameters based on real-world datasets I've used.
            %
            % Another work item on evaluation that is not strictly necessary, but
            % is beneficial: Currently doing measurements where assume each PE take
            % the same amount of time to finish computation, i.e. similar to SIMD, but
            % more realistic to assume that each PE is independent, so can also
            % measure the communication time that is waiting for message senders to
            % finish their computation. This will only require a change in the timing wrapper around the simulator,
            % not a change in the simulator itself. Therefore, I do not think it will take too long.
            %
            % Regarding the extensions, I have already completed 1.5 of them, so even
            % though I am behind on evaluation, done some ext. Reason for doing this is
            % that some of the evaluation work is affected my one of the extensions,
            % so suitable to do the extension first.
            %
        \item Work items left on evaluation:
            \begin{itemize}
                \item Generate more input graphs, using random graph library
                \item (Allow independent processing element (PE) execution in timing class)
            \end{itemize}
        \item Extensions done:
            \begin{itemize}
                % TODO: explain algorithm multiplication technique
                \item Finished generalising simulation and FoxOtto
                \item Halfway done with graph compression
            \end{itemize}
    \end{itemize}
\end{block}
\end{frame}

% So to summarise, the work remaining are
%
% Complete the two work items related to evaluation. After this, I will conduct some measurements and
% plot the results for use in the writeup. Then I will finish an extension I've already started on.
% After this, I will see how much time is left, and consider parallelising Floyd-Warshall as an additional
% extension. With the framwork for implementing parallel algorithm set up, extending the project with other
% algorithms does not require much implementation work. I am therefore a bit uncertain whether I should push
% dissertation writing start back by ~1 week in favour of further extensions, like parallelising another routing
% algorithm or implementing Cannon's algorithm.
% 
%
\begin{frame}[t]
    \frametitle{Moving forward}
    \begin{block}{Key work items left}
    \begin{itemize}
        \item Finish evaluation
            \begin{itemize}
                \item Generate random graphs of various sizes
                \item Allow independent processing element execution in timing wrapper % TODO: rephrease
                \item Do execution time measurements for various problem sizes
            \end{itemize}
        \item Finish graph compression optimisation (extension)
        \item (Parallelise Floyd-Warshall (extension))
        \item Start writing dissertation? Or more extensions?
            \begin{itemize}
                \item 24 February in timetable
            \end{itemize}
    \end{itemize}
    \end{block}
\end{frame}

\begin{frame}[t]
    \frametitle{What has been accomplished?}

    \begin{block}{Main work items}
        \begin{itemize}
            \item Data preparation script and class
            \item Multiprocessor simulation
                \begin{itemize}
                    \item Can pass \texttt{Worker} description for \texttt{PE(i, j)}
                    \item \texttt{Manager} and \texttt{MemoryController} handles
                        execution of \texttt{Worker}s and their communication
                \end{itemize}
            \item \texttt{FoxOtto} min-plus matrix multiplication
            \item Main APSP algorithm
        \end{itemize}
    \end{block}

    \begin{block}{Extra}
        \begin{itemize}
            \item Correctness tests
            \item Some evaluation done
        \end{itemize}
    \end{block}

\end{frame}

\begin{frame}[plain,c]
    \begin{center}
        \usebeamerfont*{frametitle} \usebeamercolor[fg]{frametitle} {\Huge Questions?}
    \end{center}
\end{frame}
\end{document}
