%% Based on the
%%   ACS project dissertation template
%%
%% Currently designed for printing two-sided, but if you prefer to
%% print single-sided just remove ",twoside,openright" from the
%% \documentclass[] line below.

\documentclass[a4paper,12pt,oneside,openright]{report}

% TODO: margins must be at least 2 cm
\usepackage[a4paper, margin=50pt]{geometry} 
\usepackage{hyperref}
\hypersetup{
    colorlinks,
    citecolor=red,
    linkcolor=black,
    urlcolor=black,
    hypertexnames=true
}
\usepackage{calc}
\usepackage{setspace}
\usepackage{titling}
% for pseudo code
\usepackage[linesnumbered,ruled]{algorithm2e} 
% theorem shortcuts
\usepackage{amsthm} 
\usepackage{txfonts} 
\newtheorem{definition}{Definition}


\def\authorname{Marcus Alexander Karmi September\xspace}
\def\examination{Computer Science Tripos -- Part II}
\def\authorcollege{Clare College\xspace}
\def\authoremail{maks2@cam.ac.uk}
\def\signaturedate{XX. XX. YYYY.}

% Proforma content
\def\candidatenumber{2244E}
\def\year{2022}
\def\wordcount{????}
\def\linecount{??}
\def\originator{Dr Jagdish Modi}
\def\supervisor{Dr Jagdish Modi}

% TODO: change this
\def\dissertationtitle{Advantages of Parallel Compared with Serial Computation, for Route-Planning Problems and How Increased Data Size May Effect This}
\usepackage{epsfig,graphicx,parskip,setspace,tabularx,xspace}
\usepackage{changepage}

%% START OF DOCUMENT
\begin{document}

\pagestyle{empty}
\singlespacing
% title page information
\begin{titlepage}

\onehalfspacing

%% AUTHOR NAME
\setlength{\parindent}{0pt}
\setlength{\parskip}{0pt}

% Name of author
\begingroup
\flushright
{\Large \bf \authorname\par}
\endgroup

% Centre of page, with title, examination, college and date
\vfill
\centering
{\linespread{1.1} \Huge \bf \dissertationtitle\par}

\vspace*{\stretch{0.1}}

{\huge \examination}

\vspace*{\stretch{0.1}}

{\huge \authorcollege}

\vspace*{\stretch{0.1}}

{\huge \today}
\vfill

\end{titlepage}

\newpage

\onehalfspacing
\newpage
{\Huge \bf Declaration}

\vspace{24pt}

I \authorname of \authorcollege, being a candidate for the M.Phil in
Advanced Computer Science, hereby declare that this report and the
work described in it are my own work, unaided except as may be
specified below, and that the report does not contain material that
has already been used to any substantial extent for a comparable
purpose.

\vspace{24pt}
Total word count: \wordcount

\vspace{60pt}
\textbf{Signed}:

\vspace{12pt}
\textbf{Date}:


\vfill

This dissertation is copyright \copyright 2010 \authorname.
\\
All trademarks used in this dissertation are hereby acknowledged.



\newpage
\vspace*{\fill}

%! TEX root = diss.tex
\documentclass[../diss.tex]{subfiles}
%%%%%%%%%%%%%%%%%%%%%%%%%%%%%%%%%%%%%%%%%%%%%%%%%%%%%%%%%%%%%%%%%%%%%%%%%%%%%%
% Proforma, table of contents and list of figures

\pagestyle{plain}

\newpage
\section*{Declaration}

I, \authorname of \authorcollege, being a candidate for Part II of the Computer
Science Tripos, hereby declare
that this dissertation and the work described in it are my own work,
unaided except as may be specified below, and that the dissertation
does not contain material that has already been used to any substantial
extent for a comparable purpose.

I, \authorname of \authorcollege,
am content for my dissertation to be made available to the students and staff of
the University.

\bigskip
\leftline{Signed}

\medskip
\leftline{Date: \today}

\section*{Acknowledgements}

TODO

\chapter*{Proforma}

{\large
    \begin{tabular}{p{120pt}p{311pt}}
        Candidate Number:   & \bf \candidatenumber \\
        % College:            & \bf St John's College                     \\
        Project Title:      & \bf \dissertationtitle \\
        Examination:        & \bf Computer Science Tripos -- Part II, May 2022  \\
        Word Count:         & \bf 11134\footnotemark[1]   \\
        Final line count : &  \bf 4081 (Java, \texttt{main/}) + 1796 (Java, \texttt{test/}) + 728 (Python) \\
        Project Originator: & \originator \\
        Supervisor:         & \supervisor                    \\
    \end{tabular}
}
%TC:ignore
\footnotetext[1]{This word count was computed using {\TeX}count. See
    \cref{sec:Word count} for more details.
}
\stepcounter{footnote}
%TC:endignore


\section*{Original Aims of the Project}

The aim was to implement a massively
parallel algorithm based on matrix-multiplication
for solving all-pairs shortest paths (APSP). For this, a parallel system were to
be simulated using Java. I would then evaluate the advantage of the parallel
algorithm for solving APSP, and investigate how this advantage changed with
increasing problem sizes.

\section*{Work Completed}

The project was successful. All the success criteria were met, and three
extensions were completed. I implemented an expressive simulator of
a massively parallel system with an interface that was simple to use. A
parallel APSP algorithm based on matrix multiplication was then implemented
using this interface, and was further generalised. I then extended the simulator
with sophisticated functionality for timing execution, and optimised the algorithm
with graph compression. Finally, I did a qualitative analysis on the benefit of
parallel computation where I considered different types of systems and problem
sizes.

\section*{Special Difficulties}

None

\tableofcontents

% Put them on the same pages
\listoffigures
\begingroup
\let\clearpage\relax
\let\cleardoublepage\relax
\listoftables
\listofalgorithms
\endgroup

%%%%%%%%%%%%%%%%%%%%%  Acronyms %%%%%%%%%%%%%%%%%%%%%
\newpage % TODO: Is it okay to put this in the proforma? Or put in appendix?
\section*{Acronyms and abbreviations}%
\label{sec:Acronyms and abbreviations}

% TODO: move acronyms to subfile
\begin{acronym}
    \acro{APSP}[APSP]{All-pairs shortest paths}
    \acro{PE}{\LU{P}{p}rocessing element}
    \acro{SSSP}{\LU{S}{s}ingle-source shortest paths}
    \acro{MatSquare}[MatSquare]{APSP via repeated matrix-squaring}
    \acro{SISD}[SISD]{Single-instruction-single-data}
    \acro{MISD}[MISD]{Multiple-instruction-single-data}
    \acro{SIMD}[SIMD]{Single-instruction-multiple-data}
    \acro{MIMD}[MIMD]{Multiple-instruction-multiple-data}
    \acro{GPU}{\LU{G}{g}raphics processing unit}
    \acro{CPU}{\LU{C}{c}entral processing unit}
    \acro{MPP}{\LU{M}{m}assively parallel processor}
    \acro{OOP}{\LU{O}{o}bject-\LU{O}{o}riented \LU{P}{p}rogramming}
    \acro{JVM}{\LU{J}{J}ava virtual machine}
    \acro{FPGA}{\LU{F}{f}ield-programmable gate array}
    \acro{IDE}{Integrated development environment}
    \acro{DAG}{\LU{D}{d}irected acyclic graph}
    \acro{API}{\LU{A}{a}pplication programming interface}
\end{acronym}


\singlespacing
\pagenumbering{roman}
\setcounter{page}{0}
\pagestyle{plain}
\tableofcontents
\listoffigures
\listoftables

\onehalfspacing

%% START OF MAIN TEXT

\chapter{Introduction}
\pagenumbering{arabic}
\setcounter{page}{1}

Some text
\chapter{Preparation}

\section{Some notes about the predecessor matrix}%
\label{sec:Some notes about the predecessor matrix}

Let $\odot$ represent the $(\min, +)$-matrix product.

Let $A$ be the adjacency matrix for a weighted directed acyclic graph, where all weights
are positive (cannot be zero).

\begin{definition}
    A matrix $P$ of size $n \times n$ is a \textit{predecessor matrix} if for
    every pair $i,j$, we have $d(i,j)=d(i,P(i,j))+w(P(i,j),j)$,
    where $d(i,j)$ is the distance from $i$ to $j$.
\end{definition}

This definition gives us a procedure to retrieve a shortest path between any pairs
of nodes $i$ and $j$, assuming that the two nodes are connected and that the path
is \textit{simple}, in other words does not revisit any nodes. The procedure works
as follows: Start with the pair $(i, j)$ and keep replacing $j$ with the predecessor
$P(i, j)$ until $i = j$, keeping track of the nodes $j$ that are visited.

When computing $A^{(k)}$, we want the product to give us the weight of all paths of
length $\leq 2^{k}$, not just paths of length equal to $2^k$. To achieve this, it is
necessary to set $A[i,i]:=0$ for all $i=1\dots n$. This allows paths of length
$< 2^k$ to be represented by one or more 0-weight cyclic edges. For instance, we could
have a 3-length path be represented as $a \rightarrow b \rightarrow c \rightarrow c$ in
$A^{(2)}$. However, this invalidates one of the assumptions behind the 
procedure to reconstruct paths from the predecessor matrix
$P$: The paths must be \textit{simple}. We can mitigate this problem by adding another
constraint to the predecessor matrix $P$: If there is a simple path $i \rightarrow^* j$
of length $\leq 2^k$ that consists of 1 or more edges, we have $P(i,j)\neq j$.
I will get to a proof of why this causes our procedure of reconstructing paths
to always terminate and give a simple path in a moment, if we construct $P$ using
algorithm $\ref{alg:one}$.

\begin{definition}
    A witness for $(A \odot B)[i, j]$ is the element $W[i,j]$ of the witness
    matrix. This witness is the index $k$ that gives
    $A[i,k] + B[k,j] = (A \odot B)[i,j]$ in other words, it is the argmin of
    $A[i,k] + B[k,j]$ in the $\odot$-product.
\end{definition}

With this definition, we can give our algorithm for computing the transitive closure
of a graph $G$ captured by adjacency matrix $A$ (see algorithm \ref{alg:one}).

\begin{algorithm}
    \caption{Transitive closure of graph with adjacency matrix A}\label{alg:one}
    \KwData{$A$}
    \KwResult{$T,P$}
    $A^{(0)} \gets A$

    For all $i$, $A^{(0)}[i, i] \gets 0$

    For all $i$ and $j$, $P^{(0)} \gets j$ if $A^{(0)}[i,j] = \infty, i$ otherwise


    \For{$k=1,\dots,\log n$}{
        $A^{(k)} \gets (A^{(k-1)})^2$

        $W^{(k)} \gets $ witness matrix for the product above

        \For{all pairs $i$ and $j$}{
            \eIf{$W^{(k)}[i,j]=j$}{
                $P^{(k)}[i,j] \gets P^{(k-1)}[i,j]$
            }{
                $P^{(k)}[i,j] \gets P^{(k-1)}[W^{(k)}[i,j],\;j]$
            }
        }
    }
    $T \gets A^{(\log n)}$

    $P \gets P^{(\log n)}$

    \Return{$(T, P)$}
\end{algorithm}

We now argue for why the returned predecessor matrix is correct, where \textit{correct}
means that it will cause the above-described procedure for reconstructing paths to
halt at $i=j$ for any pair $i\neq j$ if there is a simple path of non-zero length from $i$
to $j$.  We do so by induction on the $k$.

After iteration $k$, we claim $P^{(k)}$ is a correct predecessor matrix for all paths
$i \rightarrow^* j$ of length $\leq 2^k$. The base case $k=0$ follows from line 3
in our algorithm. When $k > 0$, assume there is a path of length $\leq 2^k$ from $i$
to $j$. If we let $w = W^{(k)}[i,j]$, we get by induction that there are paths
$i \rightsquigarrow w$ and $w \rightsquigarrow j$ (which are simple if $i\neq w$
and $w\neq j$, respectively) which are each of length $\leq 2^{k-1}$. Consider the
case $w=j$: This triggers the if-condition on line 8, so we use the same predecessor
as previously, which must be correct by induction. Now consider $w\neq j$: The
predecessor of path $i \rightsquigarrow j$ is the same as predecessor of path
$w \rightsquigarrow j$, which we assign on line 11.

I have written the above code for ease of reading and to make the proof simpler.
When implementing this, it is possible to perform both the matrix multiplication,
the witness matrix computation and compute the predecessor matrix, all in one go,
allowing everything to be computed in parallel. This can be done as follows:

\begin{algorithm}
    \caption{Computation done by $PE(i,j)$}\label{alg:two}
    \KwData{$i,j,n,A,W,P$}

    $a \gets \infty$

    $W[i, j] \gets j$

    \For{$w=1,\dots,n$}{
        \tcc{Prior to this, $PE(i, j)$ has received $P[w,j]$, $A[w, j]$ and $A[i,w]$ from
        the appropriate processing elements.}

        \If{$A[i,w] + A[w,j] < a$}{
            $a \gets A[i,w] + A[w,j]$

            $W[i, j] \gets w$ \tcp{The matrix $W$ is not needed here,
            but given here for completeness}

            \eIf{$w = j$}{
                $P[i, j] \gets P[i,j]$
            }{
                $P[i, j] \gets P[w, j]$
            }
        }
    }



\end{algorithm}







\chapter{Implementation}
Even more text
\chapter{Evaluation}
text
\chapter{Conclusions}
text

\end{document}
