%! TEX root = diss.tex
\documentclass[../diss.tex]{subfiles}
\chapter{Introduction}

% Clear motiviation, justing potential benefits of success
% How does work fit into areae of compsci
% related work
% Aim for 500 words here...

% intro to intro, one sentence descriving whole project; implemented simulator, then developed algo on top, tested on different graphs, random and also a large california real world dataset, tested and evaluated

% speeding up computation with parallelism:
% * stop to moore's law, but multiple processors have more agg. performance per
%   power consumption
% * many different kind of parallel systems common today
% * given a sequential algorithm to solve a problem, limit to how efficiently
%   utilize resources, compiler optimisations, limtied beneift
% * need to redesign algorithm with parallelism in mind

% why APSP:
% * pathfinding usable in many different domains, planning routes in transport
%   networks, route packets across internet to minimize some metric, planning
%   actions in AI

% something about how large benefit is when large problem sizes, but keep
% parallel system same
\section{Motivation \& aims}%
\label{sec:Motivation}

\subsection{Parallel computation}%
\label{sub:Parallel computation}

Since the 1970s, there has been a yearly exponential increase in the
number of transistors in integrated circuits, thanks to Moore's law.
% cite https://www.karlrupp.net/2018/02/42-years-of-microprocessor-trend-data/
However, around 2006, Dennard scaling started to break down, which meant that
the increase in transitory density did not directly translate to an increase in
performance per watt on a single processor
% cite https://users.cs.duke.edu/~lkw34/papers/dennard-ieee2007.pdf
As a result, performance improvements had to be achieved through other means.
The innovation of multi-core processors allowed the aggregated performance per
watt to keep increasing by combing many \acf{PE}. Today, there are a
huge variety of different parallel systems.

Designing algorithms to explore the computational resources of parallel systems
is important. To fully utilise the computational power of these new systems to
increase performance, a new kind of algorithms must be considered.  Given a
serial algorithm that solves a problem, there is a limit to how much
performance increase automatic parallelisation can achieve. Therefore, the
performance of such algorithms will not scale well with the current trends in
processor manufacturing. Instead, we should go back to the problem to be solved
and rethink an algorithm with parallel computation in mind. That way, we can
utilise the available \acp{PE} fully and increase performance to a higher
degree.

\subsection{The all-pairs shortest path problem}%
\label{sub:The all-pairs shortest path problem}

In the shortest path problem, the goal is finding the \textit{shortest path} in
a graph $G=(V,E,w)$ from a source vertex $i$ to a sink vertex $j$. In the
\ac{APSP} variant of the problem, we want to find the shortest paths between
all pairs of vertices $(i,j) \in V^2$. By working with weighted graphs, a
path-finding algorithm can be applied in many different areas. For example, it
can be used for planning routes in transport networks, route packets across the
Internet to maximise some given metric, or planning a robot's movement.

\subsection{Related work and proposed analysis}%
\label{sub:Related work}

There is a plethora of different parallel algorithms to solve route-planning
problems \cite{astar,parallelDijkstra,floydScale,scalability,experimentalAnalysis}.
However, the evaluation of the proposed or surveyed
algorithms differ greatly from paper to paper. In some research, the analysis
of how the execution time scales with the problem is mostly based on 
the algorithm's asymptotic complexity
\cite{scalability,parallelDijkstra}. \citeauthor{experimentalAnalysis} and
\citeauthor{floydScale} have analysed the performance of the algorithms through
experimental measures, but they did not run them on more than 256 \acp{PE}.
Additionally, their test graphs very dense, randomly generated graphs, not based
on real-world graphs such as road-networks.

This project aims to analyse the scaling of a parallel algorithm for solving
\ac{APSP} on real-world-like graphs, using a simulation of a parallel system
with a massive number of \acp{PE}. The project will also investigate
the benefit of parallel execution for different problem sizes and parallel
systems, considering both the number of \acp{PE} and the class of the parallel
system.


% attemts at paralleising route-planning algorithms
% A*: https://arxiv.org/abs/1708.05296
% Dijkstra \cite{parallelDijkstra}
% FloydWarshall: servlets purl one

% also many paralleisation of matrix multiplication
% berntsen, 

% also work on analysing how well these algorithm scale, but mostly based on
% models created from the asymptotic complexity of the algorithms:
% 
% theoretical predictions on performance scalability
% https://www.sciencedirect.com/science/article/pii/074373159190083L?via%3Dihub
% parallelDijkstra
% or experimental, but limited in number of processing elements
% \cite{experimentalAnalysis,floydScale}
% https://www.osti.gov/servlets/purl/1814306
% https://dl.acm.org/doi/abs/10.1145/3229710.3229730


% Use a small number of threads (< 50)
% All the way up to 256, but use dense random graphs ??? not real-world graphs?
% https://www.osti.gov/servlets/purl/1814306
% Mostly focused on asynptotic time analysis
% parallel dijkstra
% http://ilpubs.stanford.edu:8090/59/1/1994-25.pdf
% \cite{berntsen}

% Overview of the whole dissertation
% * summary of work completed
% * (structure of how worked as done, mixing prep. and implm.)
% * roadmap for dissertation, how go about presenting work, what is in diff. chaps
% * s

\section{Project structure}%
\label{sec:Project structure}


In the Preparation chapter,
% TODO: add points to go here

