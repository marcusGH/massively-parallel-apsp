\documentclass[../diss.tex]{subfiles}
%%%%%%%%%%%%%%%%%%%%%%%%%%%%%%%%%%%%%%%%%%%%%%%%%%%%%%%%%%%%%%%%%%%%%%%%%%%%%%
% Proforma, table of contents and list of figures

\pagestyle{plain}

\chapter*{Proforma}

{\large
    \begin{tabular}{ll}
        Name:               & \bf Martin Richards                       \\
        College:            & \bf St John's College                     \\
        Project Title:      & \bf How to write a dissertation in \LaTeX \\
        Examination:        & \bf Computer Science Tripos -- Part II, July 2001  \\
        Word Count:         & \bf 1587\footnotemark[1]
        (well less than the 12000 limit)  \\
        Project Originator: & Dr M.~Richards                    \\
        Supervisor:         & Dr Markus Kuhn                    \\
    \end{tabular}
}
\footnotetext[1]{This word count was computed
    by \texttt{detex diss.tex | tr -cd '0-9A-Za-z $\tt\backslash$n' | wc -w}
}
\stepcounter{footnote}


\section*{Original Aims of the Project}

To write a demonstration dissertation\footnote{A normal footnote without the
complication of being in a table.} using \LaTeX\ to save
student's time when writing their own dissertations. The dissertation
should illustrate how to use the more common \LaTeX\ constructs. It
should include pictures and diagrams to show how these can be
incorporated into the dissertation.  It should contain the entire
\LaTeX\ source of the dissertation and the makefile.  It should
explain how to construct an MSDOS disk of the dissertation in
Postscript format that can be used by the book shop for printing, and,
finally, it should have the prescribed layout and format of a diploma
dissertation.


\section*{Work Completed}

All that has been completed appears in this dissertation.

\section*{Special Difficulties}

Learning how to incorporate encapulated postscript into a \LaTeX\
document on both Ubuntu Linux and OS X.

\newpage
\section*{Declaration}

I, [Name] of [College], being a candidate for Part II of the Computer
Science Tripos [or the Diploma in Computer Science], hereby declare
that this dissertation and the work described in it are my own work,
unaided except as may be specified below, and that the dissertation
does not contain material that has already been used to any substantial
extent for a comparable purpose.

\bigskip
\leftline{Signed [signature]}

\medskip
\leftline{Date [date]}

\tableofcontents

\listoffigures

%%%%%%%%%%%%%%%%%%%%%  Acronyms %%%%%%%%%%%%%%%%%%%%%
\newpage % TODO: Is it okay to put this in the proforma? Or put in appendix?
\section*{Acronyms and abbreviations}%
\label{sec:Acronyms and abbreviations}

% TODO: move acronyms to subfile
\begin{acronym}
    \acro{APSP}[APSP]{\emph{All-Pairs Shortest Paths}}
    \acro{MatSquare}[MatSquare]{APSP via repeated matrix-squaring}
    \acro{SISD}[SISD]{Single-instruction-single-data}
    \acro{MISD}[MISD]{Multiple-instruction-single-data}
    \acro{SIMD}[SIMD]{Single-instruction-multiple-data}
    \acro{MIMD}[MIMD]{Multiple-instruction-multiple-data}
    \acro{GPU}{\LU{G}{g}raphics processing unit}
    \acro{CPU}{\LU{C}{c}entral processing unit}
    \acro{PE}{\LU{P}{p}rocessing element}
    \acro{MPP}{\LU{M}{m}assively parallel processor}
\end{acronym}

\newpage
\section*{Acknowledgements}

This document owes much to an earlier version written by Simon Moore
\cite{Moore95}.  His help, encouragement and advice was greatly
appreciated.
