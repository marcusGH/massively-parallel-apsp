%! TEX root = diss.tex
\documentclass[../diss.tex]{subfiles}
%%%%%%%%%%%%%%%%%%%%%%%%%%%%%%%%%%%%%%%%%%%%%%%%%%%%%%%%%%%%%%%%%%%%%%%%%%%%%%
% Proforma, table of contents and list of figures

\pagestyle{plain}

\newpage
\section*{Declaration}

I, \authorname of \authorcollege, being a candidate for Part II of the Computer
Science Tripos, hereby declare
that this dissertation and the work described in it are my own work,
unaided except as may be specified below, and that the dissertation
does not contain material that has already been used to any substantial
extent for a comparable purpose.

\bigskip
\leftline{Signed}

\medskip
\leftline{Date: \today}

\section*{Acknowledgements}

TODO

\chapter*{Proforma}

{\large
    \begin{tabular}{p{120pt}p{311pt}}
        Candidate Number:   & \bf \candidatenumber \\
        % College:            & \bf St John's College                     \\
        Project Title:      & \bf \dissertationtitle \\
        Examination:        & \bf Computer Science Tripos -- Part II, May 2021  \\
        Word Count:         & \bf 11134\footnotemark[1]   \\
        Final line count : &  \bf 4081 (Java, \texttt{main/}) + 1796 (Java, \texttt{test/}) + 728 (Python) \\
        Project Originator: & \originator \\
        Supervisor:         & \supervisor                    \\
    \end{tabular}
}
%TC:ignore
\footnotetext[1]{This word count was computed using {\TeX}count. See
    \cref{sec:Word count} for more details.
}
\stepcounter{footnote}
%TC:endignore


\section*{Original Aims of the Project}

The aim was to implement a massively
parallel algorithm based on matrix-multiplication
for solving all-pairs shortest paths (APSP). For this, a parallel system were to
be simulated using Java. I would then evaluate the advantage of this parallel
algorithm for solving APSP, and investigate how this advantage changed with
increasing problem sizes.

\section*{Work Completed}

The project was successful. All the success criteria were met, and three
extensions were completed. I implemented a an expressive simulator of
a massively parallel system with an interface that was simple to use. A
parallel APSP algorithm based on matrix multiplication was then implemented
using this interface, and was further generalised. I then extended the simulator
with sophisticated functionality for timing execution, and optimised the algorithm
with graph compression. Finally, I did a qualitative analysis on the benefit of
parallel computation where I considered different types of systems and problem
sizes.

\section*{Special Difficulties}

None

\tableofcontents

% Put them on the same pages
\listoffigures
\begingroup
\let\clearpage\relax
\let\cleardoublepage\relax
\listoftables
\listofalgorithms
\endgroup

%%%%%%%%%%%%%%%%%%%%%  Acronyms %%%%%%%%%%%%%%%%%%%%%
\newpage % TODO: Is it okay to put this in the proforma? Or put in appendix?
\section*{Acronyms and abbreviations}%
\label{sec:Acronyms and abbreviations}

% TODO: move acronyms to subfile
\begin{acronym}
    \acro{APSP}[APSP]{\emph{All-Pairs Shortest Paths}}
    \acro{SSSP}{\emph{Single-source shortest paths}}
    \acro{MatSquare}[MatSquare]{APSP via repeated matrix-squaring}
    \acro{SISD}[SISD]{Single-instruction-single-data}
    \acro{MISD}[MISD]{Multiple-instruction-single-data}
    \acro{SIMD}[SIMD]{Single-instruction-multiple-data}
    \acro{MIMD}[MIMD]{Multiple-instruction-multiple-data}
    \acro{GPU}{\LU{G}{g}raphics processing unit}
    \acro{CPU}{\LU{C}{c}entral processing unit}
    \acro{PE}{\LU{P}{p}rocessing element}
    \acro{MPP}{\LU{M}{m}assively parallel processor}
    \acro{OOP}{\LU{O}{o}bject-\LU{O}{o}riented \LU{P}{p}rogramming}
    \acro{JVM}{\LU{J}{J}ava virtual machine}
\end{acronym}
