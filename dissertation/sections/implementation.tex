%! TEX root = diss.tex
\documentclass[../diss.tex]{subfiles}
\chapter{Implementation}

% NOTE: Multiprocessor => shared memory. Use term "parallel computer" instead

This chapter is empty, except for the inkscape figure shown in \autoref{fig:inkscape}.

\begin{figure}
\begin{center}
    % Note on inkscape figure: Use the \showthe\textwidth (and latexmk to see output
    %   and then create inkscape documents of that width when illustrating concepts!
    \includegraphics[scale=1]{figs/inkscape-example.eps}
\end{center}
\caption{An inkscape EPS diagram}
\label{fig:inkscape}
\end{figure}

Also, the introduction should go here.

% Graph datasets
% Graph datasets {{{
\section{Graph datasets}%
\label{sec:Graph datasets}

% }}}

% Simulation of a distributed memory multiprocessor:
% * Executor service and why
% * Diagram of how work is simulated (blocks, 8 at a time)
% * Timing analysis, and how MIMD was able to be simulated:
%   * Explain wrapper, how done timing, repetition of computation, possible because of work
%     management, which gives good error bars
% Simulation of a distributed memory multiprocessor {{{
\section{Simulation of a distributed memory multiprocessor}%
\label{sec:Simulation of a distributed memory multiprocessor}

% }}}

% APSP via repeated matrix-squaring
% * Can abbriviate as MatSquare or something
% * Explain FoxOtto (explain predecessor functionality and edge-case,
%   generalized version, pseudo-code, diagram of memory movement can go in preparation)
% * Explain driver code, and its complexity?
% APSP via repeated matrix-squaring {{{
\section{APSP via repeated matrix-squaring}%
\label{sec:APSP via repeated matrix-squaring}

The \emph{\acl{MatSquare}} algorithm, which I will abbreviate as \emph{\acs{MatSquare}},
is an algorithm. \acused{MatSquare}
By using \ac{MatSquare}, we can ...
% }}}

% Graph compression:
% * The algorithm for this, and explain all the edge cases.
% * Also a section for the expected asymptotic speed-up referencing random graph generation
% Graph compression {{{
\section{Graph compression}%
\label{sec:Graph compression}

% }}}

% vim: foldmethod=marker
