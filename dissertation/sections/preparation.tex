\documentclass[../diss.tex]{subfiles}
\chapter{Preparation}

Introduction goes here.

% Parallel computing:
% * Distributed memory model and alternative
% * Communication?
% * Flynn's taxonomy
% * Evaluating a parallel algorithm, efficiency, communication-computation ratio
% Parallel computing {{{
\section{Parallel computing}%
\label{sec:Parallel computing}
Hey

% }}}

% APSP algorithm:
% * Mention known algorithms like Dijkstra, Floyd-Warshall, not very parallisable
% * MatMul highly parallisable and can be used
% * Repeated matrix squaring
% * Fox otto and Cannon's algorithm
% *
% APSP algorithm {{{
\section{APSP algorithm}%
\label{sec:APSP algorithm}
This section is empty
% }}}

% Requirements analysis
% Requirements analysis {{{
\section{Requirements analysis}%
\label{sec:Requirements analysis}

% }}}

% Choice of tools
% * Java, OOP, modularity benefits
% * GitHub, CI to check builds when push
% * JUnit4, unit testing framework for Java
% * Python for data analysis and plotting
% Choice of tools {{{
\section{Choice of tools}%
\label{sec:Choice of tools}
    
% }}}

% Starting point
% * ?
% Starting point {{{
\section{Starting point}%
\label{sec:Starting point}

% }}}


% High-level implementation overview
% * Diagram of key components and reference these in section below
% High-level implementation overview {{{
\section{High-level implementation overview}%
\label{sec:High-level implementation overview}

% }}}


% Software engineering
% Software engineering {{{
\section{Software engineering}%
\label{sec:Software engineering}

Combination of iterative development model and incremental development models,
% TODO: what is iteratively developed and what is incrementally developed?

classes, modularisation, encapsulation, immutability,
access control

documentation following JavaDocs? standard, coherent coding style used,
assertion and exceptions, unit tests for specific modules, and to test basic
interaction

% }}}

% Conclusion
% Conclusion {{{
\section{Conclusion}%
\label{sec:Conclusion}

% }}}

% vim: foldmethod=marker
