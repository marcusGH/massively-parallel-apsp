%! TEX root = diss.tex
\documentclass[../diss.tex]{subfiles}
\chapter{Preparation}

Introduction goes here.

% Parallel computing (400 words?):
% * Flynn's taxonomy
% * Distributed memory model and alternative, Communication?
% * Massively parallel systems: Talk about supercomputers, grid computing,
%   sunway etc.
% * Evaluating a parallel algorithm, efficiency, communication-computation ratio
% Parallel computing {{{
\section{Parallel computing}%
\label{sec:Parallel computing}

Parallel systems can be classified according to several different attributes.
Here, I will ...

\subsection{Flynn's taxonomy}%
\label{sub:Flynn's taxonomy}

One way to classify parallel systems, is to look at how many streams of data
there are and how many sets of instructions the computer is operating with.
% What is a "stream" of data?

\begin{itemize}
    \item \ac{SISD} -- This is the same as sequential
        execution.
    \item \ac{MISD} -- Here, multiple sets of
        instructions are executed on the same input data. There are very few
        systems using this type of parallelism.
    \item \ac{SIMD} -- With this scheme, we apply
        the same set of operations to multiple sets of data simultaneously.
    \item \ac{MIMD} -- This is the most common
        type of parallelism in hardware. Multiple sets of instructions, possibly
        different, are applied to multiple sets of data at the same time.
\end{itemize}

Parallel systems typically use either \ac{SIMD} or \ac{MIMD}, where \ac{SIMD}
is common in \acp{GPU} and \ac{MIMD} is more often seen in systems where the
computation is spread among \acp{CPU}.

%There are also hybrid categories such as SPMD, relevant for supercomputer...

\subsection{Memory and communication}%
\label{sub:Memory and communication}

We can also classify parallel systems by how the memory is laid out. In systems
that use \textit{distributed memory}, each processor has its own private memory.
If communication between processors is required, there will be an interconnection
between the \acp{PE} that allows message passing. One common approach is an
interconnect network that can be arranged in a plethora of different
\textit{topologies}. If the topology is not fully-connected, the messages need
to be routed. A common network arrangement is the 2D Lattice Mesh, where there are
$n^2$ \acp{PE} connected to each of their four orthogonal neighbours and there
are cyclic connection that wrap around the edges.
In \textit{shared memory} systems, all the processors have access to a shared
address space. Multiple  \acp{PE} can communicate by reading and writing
to the same segment of memory, and locking mechanisms can be implemented to
prevent race conditions. % Fact check?

\subsection{Something something classes MPP and others}%
\label{sub:Something something classes MPP and others}

By looking at the extent to which the system supports parallelism and at the
physical distance between the \acp{PE}, we create further classifications.
In \textit{multi-core processors}, we refer to the \acp{PE} as \textit{cores}
and have several of them on the same chip. The cores execute in \ac{MIMD}-fashion
and there are typically both local memory for each core as well as some shared
memory between cores, that can be used for communication.
In \acp{MPP}, we often have many thousands of processors, interconnected through
a specialised network. An example is the Sunway TaihuLight supercomputer
% TODO: reference....

\subsection{Evaluating performance}%
\label{sub:Evaluating performance}

A simple performance metric is the elapsed time, which works well when comparing
two different computers. However, it is not very useful when evaluating how 
efficiently
an algorithm exploits the parallelism available on a given system. For this,
the \textit{speed-up} is more useful:
\begin{equation}
    S_p = \frac{T_1}{T_p},
\end{equation}
where $T_1$ is the required time to solve the problem on a single
processor and $T_p$ is the elapsed time when executing the parallel algorithm
on $p$ processors.

Another useful metric is the \textit{parallel efficiency}, which is defined as
\begin{equation}
    \varepsilon = \frac{T_1}{p T_p} = \frac{S_p}{p}.
\end{equation}
The parallel efficiency ranges from 0 to 1, where $\varepsilon  =1$, means
we have a perfect linear speed-up where no
computation power is wasted when executing the algorithm on our parallel system.

Both of these metrics do not take into account the speed-up lost from the algorithm
having a serial part that cannot be parallelised. To incorporate this,
Amdahl's law or Gustafson's law have been used  \citep{parallelPerformance}.
However, the parallel algorithms developed in this project to solve \ac{APSP}
have a negligible serial part if we ignore the time required to read the input
and print the result. I will therefore not use these laws in evaluation.

In practice, what prevents linear speed-up is the \textit{communication cost} the
algorithm incurs while running on the parallel system. This can include both
\acp{PE} needing to idle while they wait for data to be ready or they are
waiting for some data to be transferred. % TODO: aren't these the same?
Another metric is therefore the ratio between computation and communication,
where a high ratio means we use our resources efficiently. We let
$T_p = \frac{T_{communication}+T_{computation}}{p}$, where $T_p$ is the time spent
on executing the parallel algorithm, where the sum of the costs are split up
into communication and computation costs. To execute a similar algorithm on a serial machine, we would avoid all the communication cost, but would still need to
do all the computation, so $T_1=T_{computation}$\footnote{
    This is assuming we are not using a completely different algorithm
    when solving the problem serially.
}.
We can relate this with the parallel
efficiency:
\begin{equation}
    \varepsilon = \frac{T_1}{p T_p} = \frac{T_{computation}}{T_{computation}+T_{communication}}
\end{equation}
It is also easier to measure this ratio.

% TODO: All of the above is very poorly worded. Redo this!
% DONE: write from here, about communication and computation ratio, time where
%  no useful work is done, stalling, waiting for data etc., SIMD masked computation


% }}}

% APSP algorithm (400 words?):
% * Mention known algorithms like Dijkstra, Floyd-Warshall, not very parallisable
% * MatMul highly parallisable and can be used
% * Repeated matrix squaring
% * Fox otto and Cannon's algorithm
% * Some more words here
% APSP algorithm {{{
\section{APSP algorithm}%
\label{sec:APSP algorithm}

We consider a graph $G=(V,E,w)$, where we have a set $V$ of $n$ vertices and a set
$E$ of edges and each edge $e \in E$ has a non-negative weight $w(e)$ assigned
to it. The \ac{APSP} problem concerns finding the \textit{shortest path} between
each pair of vertices.
% TODO: one more sentence about text to come...

\subsection{Parallelisation}%
\label{sub:Parallelisation}

There are many widely-used algorithms for efficiently solving the
\ac{SSSP} problem, such as Dijkstra's algorithm, Bellman-Ford and A$^*$-search.
If we have fewer than $n$ \acp{PE}, a trivial parallelisation for solving \ac{APSP}
is to run one of these algorithms on each \ac{PE}, distributing the sources.
% TODO: rephrase
However, this approach does not use our parallel resources efficiently if we have
more than $n$ \acp{PE}. To exploit this, we must either consider other algorithms
or parallelise parts of the \ac{SSSP} computation in one of these algorithms.
Parallel versions of these algorithms have been developed in the past,
% TODO: citation needed
but they incur a significant overhead. I have therefore decided to use a new
algorithm based on matrix multiplication, a highly paralleliseable task,
instead.


\subsection{Repeated matrix squaring}%
\label{sub:Repeated matrix squaring}

Matrix multiplication usually happens over the \textit{semiring}\footnote{See
appendix \autoref{sec:Abstract algebra}.} $(\mathbb{R},+,\cdot)$.
However, by replacing addition with multiplication and  multiplication by the
$\min$-operator, giving the semiring $(\mathbb{R}\cup\{\infty\},\cdot,\min)$.
This operation is known as the distance product, and by repeatedly computing it
for the weight matrix, we can find the length of shortest paths. If $W$ is
an $|V| \times |V|$ matrix, and $W_{ij}=w(e_{ij})$, the entries in $W^k$ are
distances between vertices in the graph $G$ using paths of length $k$.
If we add a self-loop to each vertex with weight 0, and compute $W^n$, the matrix
will contain the shortest-distances between all pairs of nodes that are of length
at most $n$, i.e. the shortest path (because it cannot be longer).
We can compute this matrix by squaring $W$ $\lceil \log_2 n \rceil$ times. This
% TODO: explain abbreviation better
is the idea behind our \ac{MatSquare} algorithm. We also need to modify this
algorithm to also keep track of the edges used in the shortest path, but more
details of this will be explain in \autoref{sec:APSP via repeated matrix-squaring}.
% TODO: rephrase this paragraph, it does not read well.

\subsection{Fox-Otto's technique}%
\label{sub:Fox-Otto's technique}

In \cite{fox}, they describe a technique for multiplying two matrices on a
parallel distributed-memory system, such that the amount of communication is
minimized. This is a suitable technique as we maximise our parallel efficiency
by minimising the communication. If we are computing $C=A \times B$, we
can visualise the memory movement ....

% TODO: diagram...

% }}}

% Parallel simulation (200-300 words?):
% * Having review above algorithm and hardware for parallel computing, decide
%   what "model of computation" to use.
% * Don't have access to this, and don't want to use GPU because ...
% * List all the assumptions made, e.g. message passing because FoxOtto ...
% * Why MIMD:
%    * SIMD is the most suitable for _standard_ MatMul because repeated operation
%    However, we are using a modification of that where also computation for
%    predecessor pointer, and done over min-plus semiring, so none of that
%    functionality peformed if don't branch. With SIMD, we mask instruction if have
%    branches, which wastes computation. More room for gain if assume MIMD
%    * Additionally, MIMD most common in message passing systems, and SIMD systems
%    like GPUs typically use shared-memory, not distributed.
%    * (Also possible extensions, like FW, ...?) <- not very good argument
% Parallel simulation {{{
\section{Parallel architecture assumptions}%
\label{sec:Parallel architecture assumptions}


% Note: The idea behind this section is tying everything together, **why**
%   have we made all these decisions on our parallel system...

Having reviewed some of the different classes of parallel systems, as well as
the general idea behind our \ac{APSP} algorithm, we now review the assumptions
I have made about the parallel system I will be developing an algorithm for.
Regardless of system, it should have a configurable number of \acp{PE}, and
this number can be very high as we might want to evaluate performance with
many. % TODO: rephrase

\paragraph{Memory model}%
\label{par:Memory model}

I assume a fully-distributed memory model, where each \ac{PE} has its own private
memory and no address space is shared between different \acp{PE}. To allow
coordination, a message-passing interface is also assumed. This model scales well
with a very large number of \acp{PE}, compared with shared-memory systems where
the memory system becomes a bottleneck when the number of \acp{PE} that share
the same memory are in the 1000s.

\paragraph{MIMD execution}%
\label{par:MIMD execution}
I also assume that each \ac{PE} runs independently of the others, so their
computation might not be synchronised even if they all execute the same algorithm.
I have decided to assume this model of execution over \ac{SIMD} for the
following reasons:
\begin{itemize}
    \item In distributed-memory systems with message passing, \ac{MIMD} is the
        most common, whereas \ac{SIMD} is most-often seen in shared-memory
        systems.
        % TODO: for this to make sense, add details on masking and on predecessor
        %    above in prepearation
    \item With \ac{SIMD}, if two or more \acp{PE} take different branches,
        computation is masked with \texttt{NOP}-instructions. This wastes
        computation resources, and is difficult to simulate. In standard
        matrix multiplication, there are no branches, but we are implementing
        a alternative matrix multiplication algorithm for the
        $(\mathbb{R},\cdot,\min)$ semiring, where we also have computation for
        finding the predecessors. This introduces possibly large branches into
        our inner-loop.
    \item If simulated, this model gives the programmer more flexibility...
        % TODO: really?
\end{itemize}

% TODO: work from here. What assumptions are next?

% }}}



% TODO: do this next, leading to reqAnal, no do the APSP first...

% Requirements analysis (500 words)
% Requirements analysis {{{
\section{Requirements analysis}%
\label{sec:Requirements analysis}

% }}}

% Choice of tools (200 words)
% * Java, OOP, modularity benefits
% * GitHub, CI to check builds when push
% * JUnit4, unit testing framework for Java
% * Python for data analysis and plotting
% Choice of tools {{{
\section{Choice of tools}%
\label{sec:Choice of tools}

% }}}

% Starting point (100 words)
% * ?
% Starting point {{{
\section{Starting point}%
\label{sec:Starting point}

% }}}


% High-level implementation overview (100???)
% * Diagram of key components and reference these in section below
% High-level implementation overview {{{
\section{High-level implementation overview}%
\label{sec:High-level implementation overview}

% }}}


% Software engineering (200?)
% Software engineering {{{
\section{Software engineering}%
\label{sec:Software engineering}

Combination of iterative development model and incremental development models,
% TODO: what is iteratively developed and what is incrementally developed?

classes, modularisation, encapsulation, immutability,
access control

documentation following JavaDocs? standard, coherent coding style used,
assertion and exceptions, unit tests for specific modules, and to test basic
interaction

% }}}

% Conclusion
% Conclusion {{{
\section{Conclusion}%
\label{sec:Conclusion}

% }}}

% vim: foldmethod=marker
