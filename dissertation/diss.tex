% Template for a Computer Science Tripos Part II project dissertation
\documentclass[12pt,a4paper,twoside,openright]{report}
\usepackage[pdfborder={0 0 0}]{hyperref}    % turns references into hyperlinks
\usepackage[margin=25mm]{geometry}  % adjusts page layout
\usepackage{graphicx}  % allows inclusion of PDF, PNG and JPG images
\usepackage{verbatim}
\usepackage{docmute}   % only needed to allow inclusion of proposal.tex

\raggedbottom                           % try to avoid widows and orphans
\sloppy
\clubpenalty1000%
\widowpenalty1000%

\renewcommand{\baselinestretch}{1.1}    % adjust line spacing to make
                                        % more readable

\begin{document}

\bibliographystyle{plain}


%%%%%%%%%%%%%%%%%%%%%%%%%%%%%%%%%%%%%%%%%%%%%%%%%%%%%%%%%%%%%%%%%%%%%%%%
% Title


\pagestyle{empty}

\rightline{\LARGE \textbf{Martin Richards}}

\vspace*{60mm}
\begin{center}
\Huge
\textbf{How to write a dissertation in \LaTeX} \\[5mm]
Computer Science Tripos -- Part II \\[5mm]
St John's College \\[5mm]
\today  % today's date
\end{center}

%%%%%%%%%%%%%%%%%%%%%%%%%%%%%%%%%%%%%%%%%%%%%%%%%%%%%%%%%%%%%%%%%%%%%%%%%%%%%%
% Proforma, table of contents and list of figures

\pagestyle{plain}

\chapter*{Proforma}

{\large
\begin{tabular}{ll}
Name:               & \bf Martin Richards                       \\
College:            & \bf St John's College                     \\
Project Title:      & \bf How to write a dissertation in \LaTeX \\
Examination:        & \bf Computer Science Tripos -- Part II, July 2001  \\
Word Count:         & \bf 1587\footnotemark[1]
                      (well less than the 12000 limit)  \\
Project Originator: & Dr M.~Richards                    \\
Supervisor:         & Dr Markus Kuhn                    \\ 
\end{tabular}
}
\footnotetext[1]{This word count was computed
by \texttt{detex diss.tex | tr -cd '0-9A-Za-z $\tt\backslash$n' | wc -w}
}
\stepcounter{footnote}


\section*{Original Aims of the Project}

To write a demonstration dissertation\footnote{A normal footnote without the
complication of being in a table.} using \LaTeX\ to save
student's time when writing their own dissertations. The dissertation
should illustrate how to use the more common \LaTeX\ constructs. It
should include pictures and diagrams to show how these can be
incorporated into the dissertation.  It should contain the entire
\LaTeX\ source of the dissertation and the makefile.  It should
explain how to construct an MSDOS disk of the dissertation in
Postscript format that can be used by the book shop for printing, and,
finally, it should have the prescribed layout and format of a diploma
dissertation.


\section*{Work Completed}

All that has been completed appears in this dissertation.

\section*{Special Difficulties}

Learning how to incorporate encapulated postscript into a \LaTeX\
document on both Ubuntu Linux and OS X.
 
\newpage
\section*{Declaration}

I, [Name] of [College], being a candidate for Part II of the Computer
Science Tripos [or the Diploma in Computer Science], hereby declare
that this dissertation and the work described in it are my own work,
unaided except as may be specified below, and that the dissertation
does not contain material that has already been used to any substantial
extent for a comparable purpose.

\bigskip
\leftline{Signed [signature]}

\medskip
\leftline{Date [date]}

\tableofcontents

\listoffigures

\newpage
\section*{Acknowledgements}

This document owes much to an earlier version written by Simon Moore
\cite{Moore95}.  His help, encouragement and advice was greatly 
appreciated.

%%%%%%%%%%%%%%%%%%%%%%%%%%%%%%%%%%%%%%%%%%%%%%%%%%%%%%%%%%%%%%%%%%%%%%%
% now for the chapters

\pagestyle{headings}

\chapter{Introduction}

\section{Overview of the files}

This document consists of the following files:

\begin{itemize}
\item \texttt{makefile} --- The makefile for the dissertation and
                         Project Proposal
\item \texttt{diss.tex} --- The dissertation
\item \texttt{proposal.tex}  --- The project proposal 
\item \texttt{figs} -- A directory containing diagrams and pictures
\item \texttt{refs.bib} --- The bibliography database
\end{itemize}

\section{Building the document}

This document was produced using \LaTeXe which is based upon
\LaTeX\cite{Lamport86}.  To build the document you first need to
generate \texttt{diss.aux} which, amongst other things, contains the
references used.  This if done by executing the command:

\texttt{pdflatex diss}

\noindent
Then the bibliography can be generated from \texttt{refs.bib} using:

\texttt{bibtex diss}

\noindent
Finally, to ensure all the page numbering is correct run \texttt{pdflatex}
on \texttt{diss.tex} until the \texttt{.aux} files do not change.  This
usually takes 2 more runs.

\subsection{The makefile}

To simplify the calls to \texttt{pdflatex} and \texttt{bibtex}, 
a makefile has been provided, see Appendix~\ref{makefile}. 
It provides the following facilities:

\begin{description}

\item\texttt{make} \\
 Display help information.

\item\texttt{make proposal.pdf} \\
 Format the proposal document as a PDF.

\item\texttt{make view-proposal} \\
 Run \texttt{make proposal.pdf} and then display it with a Linux PDF viewer
 (preferably ``okular'', if that is not available fall back to ``evince'').

\item\texttt{make diss.pdf} \\
 Format the dissertation document as a PDF.

\item\texttt{make count} \\
Display an estimate of the word count.

\item\texttt{make all} \\
Construct \texttt{proposal.pdf} and \texttt{diss.pdf}.

\item\texttt{make pub} \\ Make \texttt{diss.pdf}
and place it in my \texttt{public\_html} directory.

\item\texttt{make clean} \\ Delete all intermediate files except the
source files and the resulting PDFs. All these deleted files can
be reconstructed by typing \texttt{make all}.

\end{description}


\section{Counting words}

An approximate word count of the body of the dissertation may be
obtained using:

\texttt{wc diss.tex}

\noindent
Alternatively, try something like:

\verb/detex diss.tex | tr -cd '0-9A-Z a-z\n' | wc -w/


\chapter{Preparation}

This chapter is empty!


\chapter{Implementation}

\section{Verbatim text}

Verbatim text can be included using \verb|\begin{verbatim}| and
\verb|\end{verbatim}|. I normally use a slightly smaller font and
often squeeze the lines a little closer together, as in:

{\renewcommand{\baselinestretch}{0.8}\small
\begin{verbatim}
GET "libhdr"
 
GLOBAL { count:200; all  }
 
LET try(ld, row, rd) BE TEST row=all
                        THEN count := count + 1
                        ELSE { LET poss = all & ~(ld | row | rd)
                               UNTIL poss=0 DO
                               { LET p = poss & -poss
                                 poss := poss - p
                                 try(ld+p << 1, row+p, rd+p >> 1)
                               }
                             }
LET start() = VALOF
{ all := 1
  FOR i = 1 TO 12 DO
  { count := 0
    try(0, 0, 0)
    writef("Number of solutions to %i2-queens is %i5*n", i, count)
    all := 2*all + 1
  }
  RESULTIS 0
}
\end{verbatim}
}

\section{Tables}

\begin{samepage}
Here is a simple example\footnote{A footnote} of a table.

\begin{center}
\begin{tabular}{l|c|r}
Left      & Centred & Right \\
Justified &         & Justified \\[3mm]
%\hline\\%[-2mm]
First     & A       & XXX \\
Second    & AA      & XX  \\
Last      & AAA     & X   \\
\end{tabular}
\end{center}

\noindent
There is another example table in the proforma.
\end{samepage}

\section{Simple diagrams}

Simple diagrams can be written directly in \LaTeX.  For example, see
figure~\ref{latexpic1} on page~\pageref{latexpic1} and see
figure~\ref{latexpic2} on page~\pageref{latexpic2}.

\begin{figure}
\setlength{\unitlength}{1mm}
\begin{center}
\begin{picture}(125,100)
\put(0,80){\framebox(50,10){AAA}}
\put(0,60){\framebox(50,10){BBB}}
\put(0,40){\framebox(50,10){CCC}}
\put(0,20){\framebox(50,10){DDD}}
\put(0,00){\framebox(50,10){EEE}}

\put(75,80){\framebox(50,10){XXX}}
\put(75,60){\framebox(50,10){YYY}}
\put(75,40){\framebox(50,10){ZZZ}}

\put(25,80){\vector(0,-1){10}}
\put(25,60){\vector(0,-1){10}}
\put(25,50){\vector(0,1){10}}
\put(25,40){\vector(0,-1){10}}
\put(25,20){\vector(0,-1){10}}

\put(100,80){\vector(0,-1){10}}
\put(100,70){\vector(0,1){10}}
\put(100,60){\vector(0,-1){10}}
\put(100,50){\vector(0,1){10}}

\put(50,65){\vector(1,0){25}}
\put(75,65){\vector(-1,0){25}}
\end{picture}
\end{center}
\caption{A picture composed of boxes and vectors.}
\label{latexpic1}
\end{figure}

\begin{figure}
\setlength{\unitlength}{1mm}
\begin{center}

\begin{picture}(100,70)
\put(47,65){\circle{10}}
\put(45,64){abc}

\put(37,45){\circle{10}}
\put(37,51){\line(1,1){7}}
\put(35,44){def}

\put(57,25){\circle{10}}
\put(57,31){\line(-1,3){9}}
\put(57,31){\line(-3,2){15}}
\put(55,24){ghi}

\put(32,0){\framebox(10,10){A}}
\put(52,0){\framebox(10,10){B}}
\put(37,12){\line(0,1){26}}
\put(37,12){\line(2,1){15}}
\put(57,12){\line(0,2){6}}
\end{picture}

\end{center}
\caption{A diagram composed of circles, lines and boxes.}
\label{latexpic2}
\end{figure}



\section{Adding more complicated graphics}

The use of \LaTeX\ format can be tedious and it is often better to use
encapsulated postscript (EPS) or PDF to represent complicated graphics.
Figure~\ref{epsfig} and~\ref{xfig} on page \pageref{xfig} are
examples. The second figure was drawn using \texttt{xfig} and exported in
{\tt.eps} format. This is my recommended way of drawing all diagrams.


\begin{figure}[tbh]
\centerline{\includegraphics{figs/cuarms.pdf}}
\caption{Example figure using encapsulated postscript}
\label{epsfig}
\end{figure}

\begin{figure}[tbh]
\vspace{4in}
\caption{Example figure where a picture can be pasted in}
\label{pastedfig}
\end{figure}


\begin{figure}[tbh]
\centerline{\includegraphics{figs/diagram.pdf}}
\caption{Example diagram drawn using \texttt{xfig}}
\label{xfig}
\end{figure}


\chapter{Evaluation}

\section{Printing and binding}

Use a ``duplex'' laser printer that can print on both sides to print
two copies of your dissertation. Then bind them, for example using the
comb binder in the Computer Laboratory Library.

\section{Further information}

See the Unix Tools notes at

\url{http://www.cl.cam.ac.uk/teaching/current-1/UnixTools/materials.html}


\chapter{Conclusion}

I hope that this rough guide to writing a dissertation is \LaTeX\ has
been helpful and saved you time.


%%%%%%%%%%%%%%%%%%%%%%%%%%%%%%%%%%%%%%%%%%%%%%%%%%%%%%%%%%%%%%%%%%%%%
% the bibliography
\addcontentsline{toc}{chapter}{Bibliography}
\bibliography{refs}

%%%%%%%%%%%%%%%%%%%%%%%%%%%%%%%%%%%%%%%%%%%%%%%%%%%%%%%%%%%%%%%%%%%%%
% the appendices
\appendix

\chapter{Latex source}

\section{diss.tex}
{\scriptsize\verbatiminput{diss.tex}}

\section{proposal.tex}
{\scriptsize\verbatiminput{proposal.tex}}

\chapter{Makefile}

\section{makefile}\label{makefile}
{\scriptsize\verbatiminput{makefile.txt}}

\section{refs.bib}
{\scriptsize\verbatiminput{refs.bib}}

\chapter{Project Proposal}

\documentclass{article}
\usepackage{parskip,xspace}
\usepackage{geometry,array} 
\usepackage{natbib}
\usepackage[colorlinks,citecolor=red,urlcolor=blue,bookmarks=false,hypertexnames=true]{hyperref} 
\usepackage{longtable} 
\usepackage{setspace} \onehalfspacing


\def\authorname{Marcus A.\ K.\ September\xspace}
\def\authorcollege{Clare College\xspace}
\def\authoremail{maks2@cam.ac.uk}
\def\dissertationtitle{A parallel algorithm for all-pairs shortest paths that minimises data movement}
\title{
    {\large Part II Computer Science Project Proposal} \\~\\
    \dissertationtitle}
\author{
    \authorname, \authorcollege \\~\\
    Originator: Dr J. Modi \\ }

\begin{document}
\maketitle

\paragraph{Project supervisor:}%
\label{par:Project Supervisor}
Dr J. Modi

\paragraph{Director of studies:}%
\label{par:Director of Studies:}
Prof L. C. Paulson

\paragraph{Project overseers:}%
\label{par:Project Overseers:}
Prof R. Mantiuk \& Prof A. M. Pitts

\newpage

% Introduction XXX: need more work
\section*{Introduction}

The all-pairs shortest path (APSP) problem is relevant for minimising resource
costs when travelling across a road network or a railway network. There are
many known serial algorithms for solving this problem, but with a highly
parallel processor, there might be more efficient solutions. After multiplying
a graph's adjacency matrix with itself $n$ times, it is possible to find the
shortest paths of length $n$ between all pairs of nodes. This can be used to
construct an algorithm that solves APSP.  Matrix multiplication is also a
highly parallel problem, meaning there is a lot of potential gain from running
it on a multiprocessor.

Many massively parallel processors have hundreds or thousands of CPUs, each
with their own private memory. The CPUs then pass data and work to each other
using interconnect channels.  For problems where the input data is so large
that it needs to be distributed across the CPUs private memory, the communication
cost is often the performance bottleneck. Therefore, algorithms built for these
multiprocessors need to minimize the amount of memory movement.  For matrix
multiplication, this can be done by employing techniques discovered by G. C.
Fox, S. W. Otto, and L. E. Cannon \cite{fox, cannon}.

This project aims to develop an algorithm to solve APSP using matrix
multiplication.  Since I do not have access to a physical massively parallel
processor, it will be simulated in Java. I will then parallelize the matrix
multiplication step of the algorithm, and use above mentioned optimizations to
minimize the data movement. If time allows, I intend to explore whether the
advantage of parallelising APSP is still significant when the problem size is
orders of magnitude larger than the number of CPUs in the multiprocessor. The
matrix multiplication based algorithm could also be compared with a
parallelised implementation of the Floyd-Warshall algorithm, as the two are
similar in nature. Further extension work includes optimising the algorithm
through graph compression and investigating the effect of different processor
interconnect topologies.

% Starting point XXX: finished-
\section*{Starting point}%
\label{sec:Starting Point}

\begin{itemize}
    \item I have experience with writing modular, object-oriented code in Java from the
        Object-Oriented Programming course and the Further Java course.
    \item I have some knowledge of parallel programming concepts from the
        Concurrent and Distributed Systems course. I also have a bit of
        experience using these concepts in code through the Further Java
        course.
    \item I know some of the principles behind parallel processors through the Computer Design course
    \item I am familiar with some route-planning algorithms like Dijkstra's algorithm through the Algorithms course
\end{itemize}

% Resources required XXX: finished-
\section*{Resources required}%
\label{sec:Resources Required}

I will use my own computer (an Acer Nitro Laptop), running both Ubuntu Linux 18.04 LTS
and Windows 10, to both write and run all of my code. It has a quad-core CPU
(an Intel i5-8300H @ 2.30Ghz), 8 GB of RAM, and an NVIDIA GTX 1050 Mobile.  I accept full responsibility for this machine and I have made contingency plans
to protect myself against hardware and/or software failure. In case of failure, I plan to
use the machines provided by the MCS for development.

The source code, associated data, and my dissertation will be regularly version controlled
using a private \texttt{git} repository hosted by GitHub. This repository will also regularly
be copied over to the Computer Laboratory MCS server.

% Substance and Structure of the Project XXX: finish-
\section*{Substance and structure of the project}%
\label{sec:Substance and Structure of the Project}

The aim of this project is to implement a matrix multiplication based algorithm that solves
APSP, and to parallelise the matrix multiplication step to run
efficiently on a massively parallel processor with a distributed memory model. That is, the processor
will have a large number of processing elements and each of these will have their own private memory.
The core part of the project can be divided into several phases:

\paragraph{Data preparation and research}%
\label{par:Preparation research}

The project will start of with a research phase where I look into parallel
computing concepts, specifically the importance of minimizing memory movement,
and techniques for parallelising matrix multiplication \cite{cannon, fox}.
Research will be done on techniques for evaluating the performance of parallel
systems and algorithms, which will be used when planning the evaluation.
Additionally, I will plan the different software modules of my project through
use of UML diagrams.

% Easy because there's coordinates, so can just extract a random circle. Would
% be nice to plot in matplotlib!
The graphs I intend to use will be taken from road network datasets found at
\cite{road-data}. A set of graphs with a wide range of sizes is required for
rigorous evaluation of the algorithm. To acquire this, I will create a program
that extracts subsections of the datasets. The nodes in the graphs are labelled
with their longitude and latitude, so this could for instance be done by
removing all the nodes that fall outside of some arbitrarily drawn geographic circle.

\paragraph{APSP algorithm}%
\label{par:APSP Algorithm}

In this step, I will solve APSP using repeated matrix multiplication of the
graph's adjacency matrix.  The algorithm should also be able to reconstruct the
list of nodes that make up the shortest paths. I will start with a serial
matrix multiplication routine, and test the algorithm for correctness on small
graphs. Later, the matrix multiplication routine will be parallelised and optimisations
such as Fox-Otto's will be applied \cite{fox}.

\paragraph{Simulating multiprocessor and preparing evaluation}%
\label{par:Simulating Multiprocessor}

Since I do not have access to a massively parallel processor, I will simulate
one using Java's \texttt{Thread} API.  Each processing element will be
represented by a Java \texttt{Thread}, and this will be abstracted away in this
phase. Each simulated processing element will have its own private memory,
consisting of a few Java variables. The only memory movement will be processing
elements sending variables to each other or receiving data from some manager
\texttt{Thread}. I do not intend to simulate the more complicated aspects of
a processor, such as memory mapping, caches and job scheduling.

The framework I develop in this phase will also be used for my quantitative
evaluations.  One approach is to include a timer in each Java thread and take
the maximum of these times after each computation phase. By doing this for all
the computation phases of the algorithms -- which have communication phases
in-between -- I would get an estimate of the computation time. For the
communication time, I could count the number of memory transfers in the
communication phase and estimate how long this would take by using realistic
values for memory latency for existing multiprocessors.

\paragraph{Parallelising matrix multiplication}%
\label{par:Parallelising Matrix Multiplication}

In this phase I will implement Fox-Otto's algorithm using the framework
developed in the previous phase \cite{fox}. After this, the
implementation will be integrated into the APSP algorithm.

% Possible extensions XXX: finish-
\section*{Possible extensions}%
\label{sec:Possible extensions}

\begin{enumerate}
    \item Generalise the implementation of Fox-Otto's algorithm to also work when
        the number of processing elements is smaller than the problem size.
        Then move onto evaluating the advantage of parallelization as this ratio
        ( $\#~nodes~/~\#~CPUs$) increases.
    \item Optimise the APSP algorithm using graph compression techniques, such as
        removing redundant nodes.
    \item Look into different processor network topologies, and how they affect
        the communication cost of running Fox-Otto's algorithm.
    \item Parallelise Floyd-Warshall's algorithm and compare the
        computation and communication cost with the matrix multiplication-based
        APSP algorithm.
\end{enumerate}

% Success criteria XXX: finished-
\section*{Success criteria}%
\label{sec:Success criteria}

The project will be considered successful if all of the following criteria are met:
\begin{itemize}
    \item Implemented an algorithm based on matrix multiplication that can find the
        length of the shortest path between all pairs of nodes in a graph, and it is able
        to give the list of nodes that make up such paths.
    \item Parallelised the matrix multiplication routine of the algorithm to run on a simulated
        massively parallel processor, where each processing element can send data to each
        other through simulated interconnects.
    \item The parallel matrix multiplication routine is optimised to minimise the
        amount of data movement between processing elements, which is done
        by using techniques such as Fox-Otto's algorithm \cite{fox}.
    \item The evaluation of the algorithm demonstrates that parallel computation gives
        a high parallel efficiency for solving APSP
\end{itemize}

\newpage

% Timetable
\section*{Timetable}%
\label{sec:Timetable}

\begin{longtable}{m{40pt}m{75pt}m{265pt}}
    \hline
    \hline
    \textbf{Week} & \textbf{Date} & \textbf{Description} \\
    \hline
    1 -- 3 & 7 Oct - 27 Oct & % Weeks & Date range
    \textbf{Project proposal and research}
    \begin{itemize}
        \item Incorporate feedback on project proposal draft and hand in the
            final project proposal
        \item Research parallel computing concepts, parallel matrix
            multiplication algorithms like \cite{fox, cannon}, and how to evaluate
            parallel systems
        \item Write small fragments of Java code that incorporates these concepts
        \item Set up version control and sort out typesetting of dissertation
    \end{itemize}
    \textbf{Milestones}
    \begin{itemize}
        \item Project proposal handed in by October 18th
        \item GitHub repository set up
    \end{itemize} \\
    \hline
    4 -- 5 & 28 Oct - 10 Nov & % Weeks & Date range
    \textbf{Data preparation and planning}
    \begin{itemize}
        \item Create program that extracts subsection of graphs from \cite{road-data}
        \item Plan evaluation of the algorithm, using knowledge obtained in the research
            phase
        \item Plan the structure of the code, and how the components interact
    \end{itemize}
    \textbf{Milestones}
    \begin{itemize}
        \item Data extractor program written
        \item UML diagrams of core project components created
    \end{itemize} \\
    \hline
    6 -- 8 & 11 Nov - 1 Dec & % Weeks & Date range
    \textit{Due to my unit of assessment, productivity during this slot might be lower.}

    \textbf{Simulating multiprocessor I}
    \begin{itemize}
        \item Write simulation framework that allows running many CPUs in isolation,
            measuring their computation time
        \item Write code fragments to test this
    \end{itemize}
    \textbf{Simulating multiprocessor II}
    \begin{itemize}
        \item Incorporate an interconnect topology into the framework
        \item Allow the simulated CPUs in the multiprocessor to communicate by sending
            data to each other through this interconnect network
        \item Create methods to find metrics for the communication cost
        \item Write code fragments to test this
    \end{itemize}
    \textbf{Milestones}
    \begin{itemize}
        \item Simple multiprocessor simulation written
        \item Simulation extended to support threads passing each other data
        \item Quantitative computation and communication cost metrics  are produced
            when running test fragments
    \end{itemize} \\
    \hline
    9 -- 11 & 2 Dec - 22 Dec & % Weeks & Date range
    \textbf{APSP algorithm}
    \begin{itemize}
        \item Write serial matrix multiplication routine
        \item Implement algorithm to compute all pairs shortest path using matrix multiplication
        \item Make algorithm reconstruct the list of nodes in the shortest paths
        \item Test algorithm for correctness
    \end{itemize}
    \textbf{Milestones}
    \begin{itemize}
        \item APSP algorithm written
        \item Algorithm produces correct results on small graphs
    \end{itemize} \\
    \hline
    12 -- 13 & 23 Dec - 5 Jan & % Weeks & Date range
    \textit{This time slot will work as slack time if the project is behind}
    \textbf{Christmas break} \\
    \hline
    14 -- 15 & 6 Jan - 19 Jan &
    \textbf{Implement efficient parallel matrix multiplication and progress report}
    \begin{itemize}
        \item Implement Fox-Otto's algorithm, using
            the multiprocessor simulation framework
        \item Write progress report
    \end{itemize}
    \textbf{Milestones}
    \begin{itemize}
        \item Written parallel matrix multiplication algorithm
        \item Progress report submitted by Friday February 4th
    \end{itemize} \\
    \hline
    16 -- 17 & 20 Jan - 2 Feb & % Weeks & Date range
    \textbf{Evaluation}
    \begin{itemize}
        \item Evaluate the performance of the parallel APSP algorithm, and compare it with
            a theoretical serial implementation
        \item Compare the advantage of parallelism with theoretical serial implementations
            as the problem size increases relative to the number of processing
            elements (extension)
    \end{itemize}
    \textbf{Milestones}
    \begin{itemize}
        \item Theoretical comparison between parallel implementation and serial
            alternative has been made
        \item Quantitative measurement of the performance of the parallel
            implementation has been done
    \end{itemize} \\
    \hline
    18 -- 20 & 3 feb - 16 Feb & % Weeks & Date range
    
    \textit{This timeslot is dedicated to extension work. If the project is
    running behind, this slot will instead serve as slack time.}
    \textbf{Extensions}

    \begin{itemize}
        \item Generalise Cannon's or Fox-Otto's algorithm to work when the
            number of CPUs do not match the problem size
        \item Do qualitative evaluation on the algorithm's performance as the
            ratio $\#~nodes~/~\#~CPUs$ increases.
        \item Optimise the algorithm using graph compression
        \item Investigate the performance effect of different interconnect topologies
        \item Parallelise Floyd-Warshall's algorithm and compare its performance with
            the core algorithm
    \end{itemize}
    \textbf{Milestones}
    \begin{itemize}
        \item Theoretical evaluation of the advantage of parallelism as the
            ratio $\#~nodes~/~\#~CPUs$ increases.
        \item Quantitative measurements of the algorithm's performance as this ratio
            changes have been made
        \item The performance effect of graph compression has been measured quantitatively
        \item A theoretical and/or quantitative evaluation of the effect of different
            interconnect topologies have been made
        \item Written parallel implementation of Floyd-Warhsall's algorithm
    \end{itemize} \\
    \hline
    21 -- 22 & 24 Feb - 9 Mar & % Weeks & Date range
    \textbf{Dissertation writing I}
    \begin{itemize}
        \item Write first draft of dissertation
        \item Send draft to Director of Studies
    \end{itemize}
    \textbf{Milestones}
    \begin{itemize}
        \item Completed first draft of dissertation
    \end{itemize} \\
    \hline
    23 -- 24 & 10 Mar - 23 Mar & % Weeks & Date range
    \textbf{Dissertation writing II}
    \begin{itemize}
        \item Incorporate feedback from first dissertation draft, producing a second draft
    \end{itemize}
    \textbf{Milestones}
    \begin{itemize}
        \item Completed second draft of dissertation
    \end{itemize} \\
    \hline
    25 -- 26 & 24 Mar - 6 Apr & % Weeks & Date range
    \textbf{Dissertation writing III}
    \begin{itemize}
        \item Repeatedly review the dissertation
        \item Refine the layout and the diagrams
    \end{itemize}
    \textbf{Milestones}
    \begin{itemize}
        \item Submitted final dissertation by May 13th
    \end{itemize} \\
    \hline
\end{longtable}

% The timeline:
% * 10 work packages, each about a fortnight of work
% * First is preparotry work
% * Last 3 is writing dissertation
% * 5th package includes progress report due ???
% * Final deadline May 13 (but aim to finish way ahead please >_<//

% 1 Prep work
% 2 Planning
% 3 APSP algorithm
% 4 Simulate
% 5 Simulate + progress report
% 6 Implement parallel matrix multiplication technique
% 7 Evaluate
% 8  Diss -> make first draft
% 9  Diss -> go through review of first draft of diss
% 10 Diss -> review

\bibliographystyle{plain}
\bibliography{bibliography}

\end{document}


\end{document}
