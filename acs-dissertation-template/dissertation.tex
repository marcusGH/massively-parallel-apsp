%%
%% ACS project dissertation template.
%%
%% Currently designed for printing two-sided, but if you prefer to
%% print single-sided just remove ",twoside,openright" from the
%% \documentclass[] line below.
%%
%%
%%   SMH, May 2010.


\documentclass[a4paper,12pt,twoside,openright]{report}


%%
%% EDIT THE BELOW TO CUSTOMIZE
%%

\def\authorname{Marcus A.\ K.\ September\xspace}
\def\authorcollege{Clare College\xspace}
\def\authoremail{maks2@cam.ac.uk}
\def\dissertationtitle{Advantages of Parallel Compared with Serial Computation, for Route-Planning Problems and How Increased Data Size May Effect This}
\def\wordcount{Not sure yet}


\usepackage{epsfig,graphicx,parskip,setspace,tabularx,xspace}

%% START OF DOCUMENT
\begin{document}


%% FRONTMATTER (TITLE PAGE, DECLARATION, ABSTRACT, ETC)
\pagestyle{empty}
\singlespacing
% title page information
\begin{titlepage}

\onehalfspacing

%% AUTHOR NAME
\setlength{\parindent}{0pt}
\setlength{\parskip}{0pt}

% Name of author
\begingroup
\flushright
{\Large \bf \authorname\par}
\endgroup

% Centre of page, with title, examination, college and date
\vfill
\centering
{\linespread{1.1} \Huge \bf \dissertationtitle\par}

\vspace*{\stretch{0.1}}

{\huge \examination}

\vspace*{\stretch{0.1}}

{\huge \authorcollege}

\vspace*{\stretch{0.1}}

{\huge \today}
\vfill

\end{titlepage}

\newpage

\onehalfspacing
\newpage
{\Huge \bf Declaration}

\vspace{24pt}

I \authorname of \authorcollege, being a candidate for the M.Phil in
Advanced Computer Science, hereby declare that this report and the
work described in it are my own work, unaided except as may be
specified below, and that the report does not contain material that
has already been used to any substantial extent for a comparable
purpose.

\vspace{24pt}
Total word count: \wordcount

\vspace{60pt}
\textbf{Signed}:

\vspace{12pt}
\textbf{Date}:


\vfill

This dissertation is copyright \copyright 2010 \authorname.
\\
All trademarks used in this dissertation are hereby acknowledged.



\newpage
\vspace*{\fill}

\singlespacing
\newpage
{\Huge \bf Abstract}
\vspace{24pt}


This is the abstract. Write a summary of the whole thing. Make
sure it fits in one page.


\newpage
\vspace*{\fill}


\pagenumbering{roman}
\setcounter{page}{0}
\pagestyle{plain}
\tableofcontents
\listoffigures
\listoftables

\onehalfspacing

%% START OF MAIN TEXT

\chapter{Introduction}
\pagenumbering{arabic}
\setcounter{page}{1}

This is the introduction where you should introduce your work.  In
general the thing to aim for here is to describe a little bit of the
context for your work --- why did you do it (motivation), what was the
hoped-for outcome (aims) --- as well as trying to give a brief
overview of what you actually did.

It's often useful to bring forward some ``highlights'' into
this chapter (e.g.\ some particularly compelling results, or
a particularly interesting finding).

It's also traditional to give an outline of the rest of the
document, although without care this can appear formulaic
and tedious. Your call.


\chapter{Background}

A more extensive coverage of what's required to understand your
work. In general you should assume the reader has a good undergraduate
degree in computer science, but is not necessarily an expert in
the particular area you've been working on. Hence this chapter
may need to summarize some ``text book'' material.

This is not something you'd normally require in an academic paper,
and it may not be appropriate for your particular circumstances.
Indeed, in some cases it's possible to cover all of the ``background''
material either in the introduction or at appropriate places in
the rest of the dissertation.


\chapter{Related Work}

This chapter covers relevant (and typically, recent) research
which you build upon (or improve upon). There are two complementary
goals for this chapter:
\begin{enumerate}
  \item to show that you know and understand the state of the art; and
  \item to put your work in context
\end{enumerate}

Ideally you can tackle both together by providing a critique of
related work, and describing what is insufficient (and how you do
better!)

The related work chapter should usually come either near the front or
near the back of the dissertation. The advantage of the former is that
you get to build the argument for why your work is important before
presenting your solution(s) in later chapters; the advantage of the
latter is that don't have to forward reference to your solution too
much. The correct choice will depend on what you're writing up, and
your own personal preference.



\chapter{Design and Implementation}

This chapter may be called something else\ldots but in general
the idea is that you have one (or a few) ``meat'' chapters which
describe the work you did in technical detail.


\chapter{Evaluation}

For any practical projects, you should almost certainly have
some kind of evaluation, and it's often useful to separate
this out into its own chapter.


\chapter{Summary and Conclusions}

As you might imagine: summarizes the dissertation, and draws
any conclusions. Depending on the length of your work, and
how well you write, you may not need a summary here.

You will generally want to draw some conclusions, and point
to potential future work.




\appendix
\singlespacing

\bibliographystyle{unsrt}
%\bibliography{dissertation}

\end{document}
